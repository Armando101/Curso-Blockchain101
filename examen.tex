\documentclass[a4paper,12pt]{/home/armando/Documentos/Cursos/LaTeX/Plantillas/lib/pub}
%\graphicspath{ {image/} }
\usepackage{wrapfig}
\usepackage[T1]{fontenc}
\usepackage{tgadventor}

\ptcSeccion{Blockchain Academy México}
\ptcTitulo{Blockchain 101}
\ptcLogoSeccion{BAM}

\begin{document}
\putLogo
\protecoTitle
\Huge\textbf{Examen final y kahoot}\\
\normalsize
\section{Introducción a blockchain}
\begin{enumerate}
	\item ¿Por qué aprender blockchain?
	\begin{itemize}
		\item Porque está de moda
		\item Porque el bitcoin será la moneda que todos usaremos en el futuro
		\item \textbf{Porque nos ayudará a construir la siguiente generación de productos, servicios y starups.}
	\end{itemize}
	\item ¿Quiénes eran los cypherpunks?
	\begin{itemize}
		\item Un grupo de anarquistas que querían \textit{hackear} Internet.
		\item Grupo de científicos, programadores, matemáticos que de acuerdo a una necesidad económica deciden crear bitcoin.
		\item Hackers que nacieron con el fin de destrozar el gobierno.
		\item \textbf{Grupo de científicos, programadores, matemáticos que de acuerdo a una necesidad de privacidad nacen para proponer alternativas a esta problemática}
	\end{itemize}
	\item ¿Cuál de los siguientes puntos \textbf{NO} pertenece al manifiesto \textit{Cypherpunk}?.
	\begin{itemize}
		\item La privacidad es necesaria para una sociedad abierta en la era electrónica.
		\item \textbf{Privacidad == secretismo}
		\item La privacidad es la capacidad de revelarse selectivamente al mundo.
		\item La privacidad con una sociedad abierta requiere sistemas anónimos para efectuar transacciones.
	\end{itemize}
	\item El dinero está respaldado por oro que cada gobierno tiene reservado.
	\begin{itemize}
		\item Verdadero
		\item \textbf{Falso}
	\end{itemize}
	\item La criptografía \textbf{simétrica} consiste en crear dos llaves, una será pública y otra privada.
	\begin{itemize}
		\item Verdadero
		\item \textbf{Falso}
	\end{itemize}
\end{enumerate}
\section{Fundamentos de blockchain}
\begin{enumerate}
	\item Blockchain es una \textit{DLT} y una \textit{DLT} siempre es una red blockchain.
	\begin{itemize}
		\item Verdadero.
		\item \textbf{Falso}.
	\end{itemize}
	\item ¿Qué problemática resuelve Blockchain?
	\begin{itemize}
		\item La necesidad económica de la sociedad.
		\item \textbf{La necesidad de confianza que tenemos mediante un tercero para realizar una actividad económica o comercial}.
		\item La necesidad de tener una fuente de ingresos mediante bitcoins.
		\item La necesidad de comunicarnos mediante la red con cualquier persona del mundo.
	\end{itemize}
	\item Cuál de los siguientes puntos \textbf{NO} es una característica de blockchain.
	\begin{itemize}
		\item \textbf{Todo el mundo puede saber qué persona hizo cierta transacción en la red}.
		\item Es distribuido
		\item Es global: si yo tengo acceso a internet, tengo acceso a Blockchain.
		\item Es inmutable: no se puede eliminar información, todo lo que yo haga queda registrado por siempre.
	\end{itemize}
	\item ¿Cuál de las siguientes tecnologías \textbf{NO} forma parte de Blockchain.
	\begin{itemize}
		\item Criptografía.
		\item Protocolo de consenso.
		\item \textbf{Nodos}
		\item Libro contable.
	\end{itemize}
	\item Es donde se encuentras todas las transacciones que se han hecho desde el día cero en la red Blockchain.
	\begin{itemize}
		\item \textbf{Ledger}.
		\item Transacciones.
		\item Red P2P
	\end{itemize}
\end{enumerate}
\section{Tipos de blockchain}
\begin{enumerate}
	\item Blockchain 1.0 se usa principalmente para:
	\begin{itemize}
		\item Transferir bitcoins.
		\item \textbf{Enviar y recibir transacciones de valor}.
		\item Crear contratos inteligentes
		\item Crear Aplicaciones descentralizadas.
	\end{itemize}
	\item Blockchain 1.0 aproximadamente cuantas \textit{TPM} (transacciones por minuto) se pueden realizar.
	\begin{itemize}
		\item 1 TPM
		\item 25 TPM
		\item \textbf{5 TPM}
		\item 200 TPM
	\end{itemize}
	\item Los Contratos inteligentes son una característica de:
	\begin{itemize}
		\item Blockchain 1.0
		\item \textbf{Blockchain 2.0}
		\item Blockchain 3.0
	\end{itemize}
	\item La red ethereum dio paso a un nuevo termino llamado:
	\begin{itemize}
		\item Blockchain 1.0
		\item \textbf{Blockchain 2.0}
		\item Blockchain 3.0
	\end{itemize}
	\item Blockchain 3.0 aproximadamente cuantas \textit{TPM} (transacciones por minuto) se pueden realizar.
	\begin{itemize}
		\item Entre 10 y 25 TPM.
		\item Entre 100 y 250 TPM.
		\item Entre 200 y 400 TPM.
		\item \textbf{Entre 2,000 y 4,000 TPM.}
		\item Entre 20,000 y 40,000 TPM.
	\end{itemize}
\end{enumerate}
\section{Blockchain 1.0 -Bitcoin}
\begin{enumerate}
	\item ¿Qué son las criptomonedas?
	\begin{itemize}
		\item \textbf{Son un medio digital de intercambio que utiliza criptografía fuerte para asegurar las transacciones}
		\item Dinero digital
		\item Contratos inteligentes
	\end{itemize}
	\item ¿Cuánto equivale un \textit{satoshi} en \textit{BTC} (Bitcoins)?.
	\begin{itemize}
		\item \textbf{0.00000001 BTC}
		\item 0.0001 BTC
		\item 0.1 BTC
		\item 0.000000001 BTC
	\end{itemize}
	\item No hay un limite de creación de Bitcoins, esto con el objetivo de que siempre haya un incentivo para los mineros y mantener la red por siempre.
	\begin{itemize}
		\item Verdadero
		\item \textbf{Falso}
	\end{itemize}
	\item \textbf{No} es una característica de los bloques.
	\begin{itemize}
		\item HASH pointer.
		\item Estampa de tiempo
		\item Datos de transacción
		\item \textbf{Datos de las personas que hicieron la transacción.}
	\end{itemize}
	\item Un \textit{HASH} es una combinación de números y letras, la longitud de éste cambia de acuerdo a cada combinación de entrada y no existen dos \textit{hashes} iguales en el mundo.
	\begin{itemize}
		\item Verdadero
		\item \textbf{Falso}
	\end{itemize}
	\item ¿Quién es Satoshi Nakamoto?
	\begin{itemize}
		\item Una persona que decidió crear una red descentralizada para eliminar la necesidad de un tercero.
		\item \textbf{Nadie lo sabe}.
		\item Un grupo de científicos y programadores de todo el mundo.
		\item Es solo un nombre fabricado por las empresas asiáticas \textbf{Sa}msung, \textbf{Toshi}ba, \textbf{Naka}michi y \textbf{Moto}rola.
	\end{itemize}
\end{enumerate}
\section{Protocolos}
\begin{enumerate}
	\item ¿Cuántos algoritmos de consenso existen?
	\begin{itemize}
		\item 2 (PoW y PoS)
		\item \textbf{Muchos}
		\item 4
	\end{itemize}
	\item Fue el primer algoritmo de consenso que se creó.
	\begin{itemize}
		\item \textbf{PoW}
		\item PoS
		\item Casper
	\end{itemize}
	\item PoW elimina toda el cómputo pesado. Entre más bitcoins tienes en tu poder mayor es la capacidad de minado.
	\begin{itemize}
		\item Verdadero
		\item \textbf{Falso}
	\end{itemize}
	\item PoS es un sistema de consenso que se basa en la reputación o historial de transacciones. A mejor reputación, mayor probabilidad para validar bloques.
	\begin{itemize}
		\item Verdadero
		\item \textbf{Falso}
	\end{itemize}
	\item ¿Por qué son tan importantes los protocolos de consenso en una red?
	\begin{itemize}
		\item Porque evitan que se pueda duplicar una criptomoneda
		\item \textbf{Porque proporcionan los medios para que los nodos alcancen un consenso respecto a la red.}
		\item Porque evitan transacciones fraudulentas
	\end{itemize}
\end{enumerate}
\section{Blockchain 2.0 -Ethereum}
\begin{enumerate}
	\item ¿Qué es ethereum?
	\begin{itemize}
		\item Una criptomoneda.
		\item \textbf{Una plataforma que nos permite crear contratos inteligentes}.
		\item Un contrato inteligente
	\end{itemize}
	\item \textbf{No} es una característica de los contrato inteligentes
	\begin{itemize}
		\item \textbf{Funciona con inteligencia artificial.}
		\item  Se ejecutan automáticamente cuando se cumple el acuerdo.
		\item Son programas informáticos.
	\end{itemize}
	\item El video juego \textit{CryptoKitties} es un buen ejemplo de:
	\begin{itemize}
		\item Tokens
		\item Criptomonedas
		\item \textbf{Aplicaciones descentralizadas}.
	\end{itemize}
	\item Es una \textit{DApp} que busca descentralizar el internet.
	\begin{itemize}
		\item Civic
		\item Status
		\item \textbf{Substratum}
		\item Golem
	\end{itemize} 
\end{enumerate}
\end{document}
